\section{Dokumentace skriptu pro pokročilé objekty}
\subsection{Úvod}
Tato kapitola shrnuje principy implementace předposlední, tedy 4. části projektu předmětu IDS. Není zde pro nadbytečnost obsažen ER diagram, ani model případů užití, jelikož je lze nalézt v dokumentaci odevzdané dříve. Pro přehlednost je okomentován i zdrojový kód.

\subsection{Vytváření pokročilých triggerů}
Nejdříve je provedeno naplnění tabulky daty. Ta jsou vytvořena, pokud možno, co nejrůznoroději tak, aby byly pokryty všechny eventuální kombinace pokročilých dotazů.

Jelikož se projekt zabývá půjčovnou hudebních nosičů s různými alby o různých délkách skladeb, bylo vhodné vytvořit \texttt{databázový trigger} nad počítáním celkové délky alba při známé délce jednotlivých skladeb. Tento trigger má název \texttt{auto\_album\_length\_trigger}, přičemž po přidání, aktualizaci, nebo odstranění skladby z alba deklaruje čtyři pomocné proměnné, které pomáhají s následnou úpravou celkového času.

Druhý a třetí trigger kontrolují správnost zadávání dat při manipulaci s výpůjčkou nosiče. Při přidávání, nebo editaci dat v tabulce \texttt{CARRIER\_BORROW\_RECORDS} automaticky kontrolují validitu dat vrácení, resp. očekávaného vrácení a při chybném vstupu volají výjimku.




\subsection{Optimalizace dotazů}
Princip optimalizace dotazů je v tomto projektu předveden na dvou případech. Dotaz zjišťující počet vypůjčených nosičů alb v jednotlivých měsících nejdříve proběhl standardně, tedy prostým příkazem \texttt{SELECT} z tabulky \texttt{CARRIER\_BORROW\_RECORDS}. V takovém případě byla jeho cena 5 (tabulka \ref{pic1}). Následně byl přidán index přistupující ke sloupci data vypůjček téže tabulky. V takovém případě spadla jeho cena na hodnotu 3 (tabulka \ref{pic2}). Tento dotaz, vzhledem ke své minimálnosti, již nelze dále optimalizovat.

\begin{figure}[h]
    \centering
    \includegraphics[scale=0.8]{1-1.png}
    \caption{\texttt{EXPLAIN PLAN} \textbf{před} optimalizací}
    \label{pic1}
\end{figure}

\begin{figure}[h]
    \centering
    \includegraphics[scale=0.8]{1-2.png}
    \caption{\texttt{EXPLAIN PLAN} \textbf{po} optimalizací}
    \label{pic2}
\end{figure}


\newpage
Druhá optimalizace je znázorněna na dotazu vypisujícím jména zákazníků, kteří si kdy vypůjčili album \textit{The best of Waterflame}. Po spojení čtyř tabulek je při standardní cestě  cena dotazu 12 (tabulka \ref{pic3}). Po vytvoření indexu do jmenného sloupce tabulky alb spadne cena dotazu na hodnotu 11 (tabulka \ref{pic4}). Tento výsledek by šel ještě dále vylepšit přidáním indexů do tabulky \texttt{CARRIER\_COLLECTION}, ale vzhledem k malému počtu záznamů by toto vylepšení čas dále nezlepšilo.



\begin{figure}[h]
    \centering
    \includegraphics[scale=0.65]{2-1.png}
    \caption{\texttt{EXPLAIN PLAN} \textbf{před} optimalizací}
    \label{pic3}
\end{figure}


\begin{figure}[h]
    \centering
    \includegraphics[scale=0.65]{2-2.png}
    \caption{\texttt{EXPLAIN PLAN} \textbf{po} optimalizací}
    \label{pic4}
\end{figure}



\newpage
\subsection{Práva a materializovaný pohled}
Definice přístupových práv k databázovým objektům pro druhého člena týmu byly provedeny příkazy:

\medskip
\noindent
\texttt{GRANT~ALL~PRIVILEGES~ON~\dots~TO~\dots}

\medskip
\noindent
a to pro každý databázový objekt zvlášť. Řešitelům se bohužel nepodařilo tuto konstrukci minimalizovat. 

Pro demonstraci materializovaného pohledu \texttt{late\_returners} patřící prvnímu členu týmu, kde tabulky byly definované druhým členem týmu, byl vybrán dotaz na konkrétní osoby, které překračují výpůjční dobu. Toto bylo dosaženo spojením čtyř tabulek autora \texttt{xdousa00}. Pohled pak patří uživateli \texttt{xpospi0k} s příslušnými právy. Konkrétní \texttt{SELECT} následuje v kódu vzápětí,  přičemž jednoduše vybere všechny záznamy.

